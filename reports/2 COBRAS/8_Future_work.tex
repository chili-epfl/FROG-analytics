There are several constrains in our study, mainly due to the limitation of valid data. Besides, the functions defined in our system could be improved by adjusting the parameters with more precision. There are several aspects of the tool in which further work can be carried out, the main ones being:
\begin{itemize}
    \item \textbf{WindowOperation}: The code could be refactored so that the current implementation of WindowOperations is reused for computing pad heuristics in a defined time window. We could also consider reaching a balance between constant-length windows (the approach that we used) and windows that have a length proportional to each Pad's total length as, for instance, in some cases some students may work (or type) faster than others and finish the task earlier.
    \item \textbf{Pad similarity slope}: it could be interesting to discover whether the similarity slope contributes to users' collaborative behavior and try to correlate it with document equality evaluation. Based on those results, we could evaluate whether it would be useful to apply machine learning method to predict the collaborative behavior based on these values.
    \item \textbf{Collaborative behavior pattern}: Analyzing users' collaborative behavior is our main target, as we would like to be able to identify common patterns in order to identify successful collaboration strategies or even predict how much longer will the students take to finish the task. How can we find a way to represent the behavior based on all the metrics we compute is a complex task and could be the main research direction for the future work.
    \item \textbf{Paragraphs and SuperParagraphs tracking}: We implemented paragraph IDs because we were interested in tracking how specific paragraphs could be shifted down by inserting new paragraphs on top, or on how paragraphs were merged and split. However, the paragraph IDs computed seem overly complex. To be able to track specific paragraphs or superparagraphs in the future, the way how we compute this IDs should be improved so that the analysis is more accurate. 
\end{itemize}