  As stated by V. Southavilay et al., Collaborative Writing has received attention since computers have been used for word processing  \cite{southavilay2009writeproc}. The many tools available for collaborative writing make the collaboration process easier and have also modified the way in which collaborative text is produced. Particularly, in Education it has been noted that, when
  using computers, students prefer to make revisions while producing instead of after producing the text \cite{ref6}.

  The process of writing is crucial in determining the final text, hence the interest in analyzing this process in Education. Learning about the process of collaboratively writing a text can help us understand different aspects about the roles taken by collaborators (big text additions, small edits...) or how the general structure of the document changes in time (list-like paragraphs, regular paragraphs with a similar length, big semantic differences between paragraphs...), among others.

  This information can be used to identify the practices that can potentially lead to a good quality text - and the practices that may decrease the quality of a text. Defining the quality of a text is not in the scope of this project, as we focus on defining the heuristics that allow to correctly capture the relevant information about the writing process.

  It would also be useful to be able to do some predictions before the task is finished, using the information gathered about the writing process up until that moment, as a tool for teachers to know how well students are collaborating in time for correcting potential mistakes.

  COBRAS (COllaBorative pRocess Analysis with Semantics) allows to learn about collaboratively written texts by looking at the process of writing. The `S' in the name has been added because one of the characteristics of the tool is the use of Semantic Analysis.
  
  The tool and the data used for the current report have been previously used to obtain the results presented in the submission to the International Conference on Computer Supported Collaborative Learning (CSCL 2019) \cite{cscl19}.