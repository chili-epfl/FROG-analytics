Compared with previous tools we were based on, we created \texttt{WindowOperation} in order to apply semantic analysis on our tool, and a better time interval will help us to analyze users' data easier. In addition, we tested several pre-trained models and found that sent2vec model has the best performance in our dataset. We can use it to analyze the similarity between \texttt{SuperParagraph} pairs to see if they are related, or apply it to different authors' \texttt{WindowOperations}. We can then obtain the slope of linear model that fits the similarity of authors' \texttt{WindowOperations} for hints on the collaborators' writing behavior.  

Regarding the heuristics implemented for the pad we could see an improvement by redefining the scores that were introduced previously so that they considered only the operations in the last window, as the latter operations are no longer averaged over the total pad and we can see their effect more clearly.

We defined some new measurements regarding the lengths of the \texttt{Paragraphs} and \texttt{SuperParagraphs} (that usually correspond with text lines and text paragraphs, respectively), but with the current timewindow implementation we were not able to reach any conclusion. However, we saw that these measures could potentially give us some interesting insights on how the pad's structure changes in time, suggesting that it may be worth it to explore them further.